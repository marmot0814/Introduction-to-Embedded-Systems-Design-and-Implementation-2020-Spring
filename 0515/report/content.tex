\title{2020/05/15 Discussion}
\author{0616014 楊政道}
\maketitle
\thispagestyle{fancy}
\section{Discussion1}
\subsection{Advertise an URL with your student ID.}
\paragraph{}
推播的網址\texttt{http://tw.0616014}, 轉成Eddystone message的值為
\begin{center}
 \begin{tabular}{||c | c||}
 \hline
 數值(16進位) & 網址\\ [0.5ex]
 \hline\hline
 02 & http:// \\
 \hline
 74 & t \\
 \hline
 77 & w \\
 \hline
 2e & . \\
 \hline
 30 & 0 \\
 \hline
 36 & 6 \\
 \hline
 31 & 1 \\
 \hline
 36 & 6 \\
 \hline
 30 & 0 \\
 \hline
 31 & 1 \\
 \hline
 34 & 4 \\
 \hline
\end{tabular}
\end{center}
\section{Discussion2}
\subsection{What is the maximum value of major and minor ID? How to calculate this value?}
\paragraph{}
major ID和minor ID都用2 bytes的空間來儲存, 2 bytes為16個bits, 所以最大的ID value為$2^{16} - 1 = 65535$。
\section{Discussion3}
\subsection{If we put a lot of beacons, can we get a more precise result?}
\paragraph{}
可以, 原理和GPS很像, 同時間最少只要有三個距離資料就可以推算出自己的三維座標位置(三個球體的交點)。
